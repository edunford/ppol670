% Options for packages loaded elsewhere
\PassOptionsToPackage{unicode}{hyperref}
\PassOptionsToPackage{hyphens}{url}
%
\documentclass[
  12pt,
]{article}
\usepackage{lmodern}
\usepackage{amssymb,amsmath}
\usepackage{ifxetex,ifluatex}
\ifnum 0\ifxetex 1\fi\ifluatex 1\fi=0 % if pdftex
  \usepackage[T1]{fontenc}
  \usepackage[utf8]{inputenc}
  \usepackage{textcomp} % provide euro and other symbols
\else % if luatex or xetex
  \usepackage{unicode-math}
  \defaultfontfeatures{Scale=MatchLowercase}
  \defaultfontfeatures[\rmfamily]{Ligatures=TeX,Scale=1}
\fi
% Use upquote if available, for straight quotes in verbatim environments
\IfFileExists{upquote.sty}{\usepackage{upquote}}{}
\IfFileExists{microtype.sty}{% use microtype if available
  \usepackage[]{microtype}
  \UseMicrotypeSet[protrusion]{basicmath} % disable protrusion for tt fonts
}{}
\makeatletter
\@ifundefined{KOMAClassName}{% if non-KOMA class
  \IfFileExists{parskip.sty}{%
    \usepackage{parskip}
  }{% else
    \setlength{\parindent}{0pt}
    \setlength{\parskip}{6pt plus 2pt minus 1pt}}
}{% if KOMA class
  \KOMAoptions{parskip=half}}
\makeatother
\usepackage{xcolor}
\IfFileExists{xurl.sty}{\usepackage{xurl}}{} % add URL line breaks if available
\IfFileExists{bookmark.sty}{\usepackage{bookmark}}{\usepackage{hyperref}}
\hypersetup{
  hidelinks,
  pdfcreator={LaTeX via pandoc}}
\urlstyle{same} % disable monospaced font for URLs
\usepackage[margin=1in]{geometry}
\usepackage{longtable,booktabs}
% Correct order of tables after \paragraph or \subparagraph
\usepackage{etoolbox}
\makeatletter
\patchcmd\longtable{\par}{\if@noskipsec\mbox{}\fi\par}{}{}
\makeatother
% Allow footnotes in longtable head/foot
\IfFileExists{footnotehyper.sty}{\usepackage{footnotehyper}}{\usepackage{footnote}}
\makesavenoteenv{longtable}
\usepackage{graphicx,grffile}
\makeatletter
\def\maxwidth{\ifdim\Gin@nat@width>\linewidth\linewidth\else\Gin@nat@width\fi}
\def\maxheight{\ifdim\Gin@nat@height>\textheight\textheight\else\Gin@nat@height\fi}
\makeatother
% Scale images if necessary, so that they will not overflow the page
% margins by default, and it is still possible to overwrite the defaults
% using explicit options in \includegraphics[width, height, ...]{}
\setkeys{Gin}{width=\maxwidth,height=\maxheight,keepaspectratio}
% Set default figure placement to htbp
\makeatletter
\def\fps@figure{htbp}
\makeatother
\setlength{\emergencystretch}{3em} % prevent overfull lines
\providecommand{\tightlist}{%
  \setlength{\itemsep}{0pt}\setlength{\parskip}{0pt}}
\setcounter{secnumdepth}{-\maxdimen} % remove section numbering
\usepackage{booktabs}
\usepackage{longtable}
\usepackage{array}
\usepackage{multirow}
\usepackage{wrapfig}
\usepackage{float}
\usepackage{colortbl}
\usepackage{pdflscape}
\usepackage{tabu}
\usepackage{threeparttable}
\usepackage{threeparttablex}
\usepackage[normalem]{ulem}
\usepackage{makecell}
\usepackage{xcolor}

\author{}
\date{\vspace{-2.5em}}

\begin{document}

\begin{center}
 

\huge \textit{PPOL 670}\\
\huge\textbf{Introduction to Data Science}\\

\Large Spring 2021
\end{center}

\hypertarget{instructor}{%
\section{Instructor}\label{instructor}}

\textbf{Professor}: Eric Dunford, Ph.D.

\begin{itemize}
\tightlist
\item
  \textbf{Office}: Bedroom (formerly 404 Old North)
\item
  \textbf{Office Hours}: Wednesdays 9am to 11am (EST)
\item
  \textbf{Email}:
  \href{mailto:eric.dunford@georgetown.edu}{\nolinkurl{eric.dunford@georgetown.edu}}
\item
  \textbf{Pronouns}: he/him
\end{itemize}

\textbf{Teaching Assistant}: Madeline (Maddie) Pickens

\begin{itemize}
\tightlist
\item
  \textbf{Office Hours}: by appointment online
\item
  \textbf{Email}:
  \href{mailto:mp1595@georgetown.edu}{\nolinkurl{mp1595@georgetown.edu}}
\item
  \textbf{Pronouns}: she/her
\end{itemize}

\textbf{Class Website}: \url{www.ericdunford.com/ppol670}

\begin{center}\rule{0.5\linewidth}{0.5pt}\end{center}

\hypertarget{course-description}{%
\section{Course Description}\label{course-description}}

This course teaches students how to synthesize disparate, possibly
unstructured data in order to draw meaningful insights from data. Topics
covered include fundamentals of functional programming in R, literate
programming, data wrangling, data visualization, data extraction (via
web scraping and APIs), text analysis, and machine learning methods. The
course aims to offer students a practical toolkit for data exploration.
The objective of the course is to equip students with the skills to
incorporate data into their decision-making and analysis. No prior
programming experience is assumed, but prior statistics training is
required.

\hypertarget{time-and-location}{%
\section{Time and location}\label{time-and-location}}

Classes will be held \textbf{\emph{virtually}} on \textbf{Tuesdays} from
\textbf{\emph{6:30 pm to 9:00pm}}:

\begin{itemize}
\tightlist
\item
  January 26
\item
  February 2, 9, 16, 23
\item
  March 2, 9, 16, 23
\item
  April 6, 13, 20, 27
\item
  May 4
\end{itemize}

Holidays/Breaks/Away (No class):

\begin{itemize}
\tightlist
\item
  March 30 (Spring Break)
\end{itemize}

\hypertarget{asynchronous-synchronous-lectures}{%
\subsection{Asynchronous \& Synchronous
Lectures}\label{asynchronous-synchronous-lectures}}

The lecture will be broken up into \emph{synchronous} and
\emph{asynchronous} components.

\begin{itemize}
\tightlist
\item
  The \textbf{\emph{asynchronous components}} will cover the main
  concepts of the lecture. These materials will take the form of
  embedded videos in class lecture notes on the course website. Students
  are required to review this content along with the lecture notes and
  readings prior to the start of class. \textbf{\emph{Asynchronous
  materials will be made available a \textbf{week prior} to the
  scheduled lecture date}.}
\item
  The \textbf{\emph{synchronous component}} will take place at the
  scheduled class time and will involve active coding walkthrough,
  breakout group sessions, and questions. The aim of the synchronous
  class time is to reinforce the concepts covered in the asynchronous
  lecture materials. Thus, it is imperative that students complete the
  asynchronous material \emph{prior to the start of the synchronous
  lecture}.
\end{itemize}

Note that this class is scheduled to meet weekly for 2.5 hours. I will
do my best to ensure that the asynchronous and synchronous material in
combination does not exceed 2.5 hours weekly. Put differently, students
will not be required to commit more than 2.5 hours to lecture. This does
not include readings, homework and/or coding discussions; rather,
bifurcating lecture materials into synchronous and asynchronous
components is necessary when learning virtually. Zoom fatigue is real,
and lectures that exceed an 1.5 hours are not effective. When we do meet
in-person, five minute breaks will be taken approximately every 40
minutes.

\textbf{All synchronous lecture material will be recorded and stored on
the class Canvas site.} Students who are unable to attend the
synchronous lecture will be able to review the materials covered in
class at a future date.

\textbf{For students attending class from afar} (i.e.~in time zones more
than 4 hours off Eastern standard time), participating in the
synchronous lecture component may not be a viable option. Please let the
professor know if you're planning on attending the course from afar.
These students will not be required to attend synchronous components of
the lecture. \emph{It is the students responsibility to review all
lecture materials and to keep pace with the course.}

\hypertarget{virtual-classroom}{%
\subsection{Virtual Classroom}\label{virtual-classroom}}

We will use \href{https://zoom.us/download}{\textbf{Zoom}} (a
web-conferencing platform) to hold class each week. Class will meet at
its regularly scheduled time each week for synchronous lectures. If you
do not have Zoom, you can download it
\href{https://zoom.us/download}{\textbf{here}} prior to the start of
class.

A link for the synchronous component of the weekly lecture along with a
link for virtual office hours is posted on the course website and
Canvas. Students will use this link to access the live Zoom call for
lecture.

\textbf{\emph{If the link breaks or does not function properly, please
check the \texttt{\#general} channel on Slack for information regarding
the new link. If there is no message regarding a new link, please
contact the professor and/or TA via Slack.}} All synchronous lecture
material will be recorded.

\hypertarget{course-objectives}{%
\section{Course Objectives}\label{course-objectives}}

This course focuses on providing students with an applied knowledge of
the \texttt{R} programming environment while placing emphasis on
developing a practical data science toolkit that students can implement
quickly and efficiently. To this end, the course takes a `Tidyverse'
approach to \texttt{R} programming, which provides users an intuitive
grammar for data manipulation and visualization. The goal is to
establish a practical toolkit for analysis in \texttt{R} without getting
too bogged down in the nuts and bolts of functional programming.

\begin{enumerate}
\def\labelenumi{\arabic{enumi}.}
\item
  Understand the basics of programming in \texttt{R} with emphasis on
  the ``tidy'' ecosystem of packages.
\item
  Learn how to wrangle (prepare and clean) different types of data.
\item
  Learn to identify and visualize important trends and findings.
\item
  Learn to extract and process data from unstructured sources, such as
  the web and/or text.
\item
  Learn to use statistical learning approaches to effectively explore
  and ask questions from data.
\item
  Learn how to query online resources to find answers to resolve
  coding-related errors/inquiries.
\end{enumerate}

\hypertarget{pre-requisites}{%
\section{Pre-Requisites}\label{pre-requisites}}

\begin{itemize}
\tightlist
\item
  \textbf{\emph{Required}}: PPOL501/531 - Statistical Methods for Policy
  Analysis (or an equivalent course)
\item
  \textbf{\emph{Preferred}}: PPOL502/532 - Regression Methods for Policy
  Analysis (or an equivalent course)
\end{itemize}

\hypertarget{required-materials}{%
\section{Required Materials}\label{required-materials}}

\textbf{Readings}: We will rely primarily on the following texts for
this course.

\begin{itemize}
\item
  \textbf{Wickham, H., \& Grolemund, G. (2016). ``R for data science:
  import, tidy, transform, visualize, and model data''. \emph{O'Reilly
  Media, Inc.}}.

  \begin{itemize}
  \tightlist
  \item
    In an effort to keep costs as low as possible, we'll resort to the
    \href{https://r4ds.had.co.nz/}{online presentation} of these
    materials. That said, many students find it useful to have a hard
    copy of the book materials. I strongly encourage students to
    purchase this book. It will serve as a valuable reference both
    during the semester and into the future.
  \end{itemize}
\item
  \textbf{James, G., Witten, D., Hastie, T., \& Tibshirani, R. (2013).
  ``An Introduction to Statistical Learning: with Applications in R''.
  \emph{New York: springer}.}
\item
  \textbf{\emph{Additional readings will be posted for each class and
  can be found on the course website}}. Most reading material is open
  source and available via a link on the reading list, otherwise it can
  be found on Canvas.
\end{itemize}

\textbf{Class Website}: A class website (www.ericdunford.com/ppol670)
will be used throughout the course and should be checked on a regular
basis for lecture materials and required readings.

\textbf{Class Slack Channel}: The class also has a dedicated slack
channel (ppol670-spring-2021). The channel serves as an open forum to
discuss, collaborate, pose problems/questions, and offer solutions.
Students are encouraged to pose any questions they have there as this
will provide the professor and TA the means of answering the question so
that all can see the response. If you're unfamiliar with Slack, please
consult the following start-up tutorial
(\url{https://get.slack.help/hc/en-us/articles/218080037-Getting-started-for-new-members}).
Please follow the
\href{https://join.slack.com/t/ppol670-spring-2021/shared_invite/zt-ku5h37xj-rN4y~pXDnQeV4YFNXsf9pw}{\textbf{\emph{invite
link}}} to be added to the Slack channel.

\textbf{Canvas}: A Canvas site
(\textbf{\url{http://canvas.georgetown.edu}}) will be used periodically
throughout the course and should be checked on a regular basis. All
assignments will be posted on Canvas; they will not be distributed in
class or by e-mail. Support for Canvas is available at (202) 687-4949

\textbf{Computing}: Programming task for in-class activities and
assignments will be conducted using \texttt{R}. Students are strongly
encouraged to utilize
\href{https://www.rstudio.com/products/rstudio/download/}{Rstudio},
which offers an accessible and widely-utilized graphical user interface
for programming in \texttt{R}.

\textbf{NOTE: In-class activities will include programming in
\texttt{R}. If you do not have access to a laptop on which you can
install \texttt{R} and \texttt{Rstudio}, please contact the professor
and/or TA for assistance.}

\hypertarget{course-requirements}{%
\section{Course Requirements}\label{course-requirements}}

\begin{longtable}[]{@{}lc@{}}
\toprule
\textbf{Assignment} & \textbf{Percentage of Grade}\tabularnewline
\midrule
\endhead
Problem sets & 50\%\tabularnewline
Final Project & 50\%\tabularnewline
\bottomrule
\end{longtable}

\begin{quote}
\emph{Note that the grades on Canvas are not weighted, and thus, may not
accurately reflect a student's final grade.}
\end{quote}

\textbf{Problem Sets} (50\%): Students will be assigned five problem
sets. While you are encouraged to discuss the problem sets with your
peers and/or consult online resources, \textbf{the finished product must
be your own work}. Problem sets are due on the date and time posted on
Canvas and must be submitted on Canvas. Late assignments will be
penalized a letter grade for every day they are overdue.

All problem sets must be submitted as \texttt{.html} files with clean,
readable code chunks using \texttt{RMarkdown}. Along with the
\texttt{.html}, student's must submit a \texttt{.zip} file containing
the \texttt{.rmd} file they used to knit the \texttt{.html} and the data
used to complete the assignment. The \texttt{.rmd} file should be
completely reproducible and contain no machine specific information
(e.g.~a file path). All assignment submissions must adhere to the
following guidelines:

\begin{itemize}
\item
  \begin{enumerate}
  \def\labelenumi{(\roman{enumi})}
  \tightlist
  \item
    all code must run;
  \end{enumerate}
\item
  \begin{enumerate}
  \def\labelenumi{(\roman{enumi})}
  \setcounter{enumi}{1}
  \tightlist
  \item
    solutions should be readable
  \end{enumerate}

  \begin{itemize}
  \tightlist
  \item
    Code should be thoroughly commented (the Professor/TA should be able
    to understand the code's purpose by reading the comment),
  \item
    Coding solutions should be broken up into individual code chunks,
    not clumped together into one large code chunk (See examples in
    class or reach out to the TA/Professor if this is unclear),
  \end{itemize}
\item
  \begin{enumerate}
  \def\labelenumi{(\roman{enumi})}
  \setcounter{enumi}{2}
  \tightlist
  \item
    Non-coding responses should all be written in Markdown and should
    contain no grammatical or spelling errors;
  \end{enumerate}
\item
  \begin{enumerate}
  \def\labelenumi{(\roman{enumi})}
  \setcounter{enumi}{3}
  \tightlist
  \item
    All programming solutions should employ concepts learned during the
    course. Specifically, students must use \texttt{tidyverse} solutions
    learned in class, over base \texttt{R} solutions pulled from the
    internet.
  \end{enumerate}
\end{itemize}

The follow schedule lays out when each assignment will be assigned and
due.

\begin{longtable}[]{@{}lll@{}}
\toprule
Assignment & Date Assigned & Date Due\tabularnewline
\midrule
\endhead
No.~1 & February 16 & February 23\tabularnewline
No.~2 & March 2 & March 9\tabularnewline
No.~3 & March 16 & March 23\tabularnewline
No.~4 & April 6 & April 13\tabularnewline
No.~5 & April 20 & April 27\tabularnewline
\bottomrule
\end{longtable}

\textbf{Final Project} (50\%): Data science is an applied field and
therefore, it is important that you understand how to conduct a complete
analysis from collecting data, to cleaning and analyzing it, to
presenting your findings. Toward the end of the semester, you will
complete an independent data science project, \emph{applying concepts
learned throughout the course}. The project is composed of three parts:
a 500 words (2-page) project proposal, an in-class presentation, and a
3000 words (12-page) project report. Due dates and breakdowns for the
project are as follows:

\begin{longtable}[]{@{}llll@{}}
\toprule
\textbf{Requirement} & \textbf{Due} & \textbf{Length} &
\textbf{Percentage}\tabularnewline
\midrule
\endhead
Project Proposal & April 6 & 500 words & 5\%\tabularnewline
Presentation & May 4 & 7 minutes & 10\%\tabularnewline
Project Report & May 18 & 3000 words & 35\%\tabularnewline
\bottomrule
\end{longtable}

Details regarding each aspect of the project will be posted on the
course website leading up to the first due date (i.e.~the Project
Proposal). Until then, we will not discuss the project in class. The
reason for this is that students need to reach a basic level of data
competency before thinking through a project idea. Thus, discussion of
the final project and the development of a project proposal will align
with the final portion of the class; once we've broadly covered most of
the fundamental data topics covered in this course.

\hypertarget{grading}{%
\section{Grading}\label{grading}}

Course grades will be determined according to the following scale:

\begin{longtable}[]{@{}ll@{}}
\toprule
Letter & Range\tabularnewline
\midrule
\endhead
A & 95\% -- 100\%\tabularnewline
A- & 91\% -- 94\%\tabularnewline
B+ & 87\% -- 90\%\tabularnewline
B & 84\% -- 86\%\tabularnewline
B- & 80\% -- 83\%\tabularnewline
C & 70\% -- 79\%\tabularnewline
F & \textless{} 70\%\tabularnewline
\bottomrule
\end{longtable}

\hypertarget{managing-the-workload-how-to-succeed-in-this-course}{%
\section{Managing the Workload: How to Succeed in this
Course}\label{managing-the-workload-how-to-succeed-in-this-course}}

\begin{itemize}
\item
  \textbf{Come Prepared.}

  \begin{itemize}
  \item
    Do the readings. Think about the readings on their own terms, but
    also in terms of how the concepts apply to things you're interested
    in.
  \item
    As this class is quite hands-on, it is expected that students bring
    their computers to class to partake in computational activities.
    Moreover, students should have all relevant software up and running
    on their machines.
  \end{itemize}
\item
  \textbf{Ask Questions.}

  \begin{itemize}
  \tightlist
  \item
    Formulating a question helps you engage with the material much more
    deeply. If you have a question, it's almost certain that others do
    too; asking a question will not only help yourself, but you will
    help others. Most importantly, asking questions helps keep the class
    on track. If there are lots of questions, we'll slow down and get
    things figured out. If there are few questions, we'll charge ahead.
  \end{itemize}
\item
  \textbf{Collaborate.}

  \begin{itemize}
  \item
    Work in groups, but do so wisely. Collaboration is the greatest
    source of creativity and innovation. Better yet, working with
    classmates is a great way to learn from each other. Often,
    classmates will have some way of explaining things that clicks for
    you, and, more often than not, the act of explaining something to
    someone else will make things click for you. This only works,
    though, if you prepare by yourself first. If you show up and wait
    for classmates to do the work, you can probably muddle through the
    homeworks, but you'll have trouble participating in classes and may
    fall behind as the material we cover cumulates and needs to be
    understood at each step.
  \item
    collaboration should not result in verbatim submissions (e.g.~no
    copy cats). As everyone writes code following their own unique
    logic, the chance of identical submissions is unlikely and easily
    detectable. Non-unique code will be penalized.
  \item
    Finally, utilize \textbf{the class Slack channel} to post any
    questions, insights, coding problems and concerns. The channel will
    offer an open forum to communicate, collaborate, and collectively
    problem solve.
  \end{itemize}
\item
  \textbf{Start homeworks early.}

  \begin{itemize}
  \tightlist
  \item
    Sometimes the data doesn't cooperate, or there is an error in your
    code that will take you awhile to figure out and debug. You don't
    want to find this out at 11pm the night before the homework is due.
    Also, the more you are doing homeworks, the more you will be able to
    follow the lectures.
  \end{itemize}
\item
  \textbf{Try doing it the hard way.}

  \begin{itemize}
  \tightlist
  \item
    A core factor in the success of a data scientist is being able to
    explain how an algorithm or analysis was constructed, not just use
    software. In this class, where possible, build from scratch rather
    than an overly convenient library. This will allow you to become
    more creative down the line.
  \end{itemize}
\end{itemize}

\hypertarget{course-policies}{%
\section{Course Policies}\label{course-policies}}

\hypertarget{participation}{%
\subsubsection{Participation}\label{participation}}

Participation is required in this course. I define participation as:

\begin{itemize}
\tightlist
\item
  Attending synchronous lecture components over Zoom.
\item
  Completing the readings and asynchronous materials prior to the
  synchronous lecture.
\item
  Asking questions and participating in class.
\item
  During synchronous lectures, cameras are active at all times.
\item
  Paying attention to the professor during lecture
\item
  Engage in break-out group discussions when assigned.
\item
  Responding to questions asked during synchronous sessions.
\end{itemize}

I reserve the right to deduct points from students final grade who are
not participating as expected.

\hypertarget{communication}{%
\subsubsection{Communication}\label{communication}}

\begin{itemize}
\item
  For private questions concerning the class, email is the preferred
  method of communication. All email messages must originate from your
  Georgetown University email account(s). Please use a professional
  salutation, proper spelling and grammar, and patience in waiting for a
  response. The professor reserves the right to not respond to emails
  that are drafted inappropriately. \textbf{\emph{Please email the
  professor and the TA directly rather than through the Canvas messaging
  system.}} Emails sent through Canvas will be ignored.
\item
  For general, class-relevant questions, \texttt{Slack} is the preferred
  method of communication. Please use the general or the relevant
  channel for these questions.
\item
  I will respond to all emails/slack questions within 24 hours of being
  sent during a weekday. I will not respond to emails/slack sent late
  Friday (after 5PM) or during the weekend until Monday (9AM). Please
  plan accordingly if you have questions regarding current or upcoming
  assignments. Please address the professor and TA by their last name
  unless stated otherwise.
\end{itemize}

\hypertarget{electronic-devices}{%
\subsubsection{Electronic Devices}\label{electronic-devices}}

The use of laptops, tablets, or other mobile devices is permitted
\emph{only for class-related work}. Audio and video recording is not
allowed unless prior approval is given by the professor. Please mute all
electronic devices during class.

\hypertarget{assignments-and-late-work}{%
\subsubsection{Assignments and Late
Work}\label{assignments-and-late-work}}

Assignments should be clear, legible, and submitted in the required
format. Writing assignments will be graded on the basis of content,
logic, analysis, mechanics, organization, and research. Due dates for
all assignments will be posted on Canvas and are non-negotiable.
Exceptions to this policy will be made only under extremely unusual
circumstances and will require valid documentation from the student.
\textbf{\emph{Late problem sets will be penalized a letter grade per
day.}}

\hypertarget{proof-of-diligent-debugging}{%
\subsubsection{Proof of Diligent
Debugging}\label{proof-of-diligent-debugging}}

When reaching out to the professor or teaching assistant regarding a
technical question, error, or issue you \textbf{\emph{must}} demonstrate
that you made a good faith effort to debugging/isolate your problem
prior to reaching out. In as concise a way as possible, send a record of
what you tried to do along with a reproducible example emulating the
error. (See the materials for Week 3 on how to generate a reproducible
example using \texttt{reprex} and \texttt{datapasta}). As software is
continually being refined in data science and new approaches continually
emerge and changing, learning how to frame your question and find a
similar solution online is a key tool for success in this domain. If you
make a diligent effort beforehand to solve your problem, we will do the
same in trying to help you figure out a solution. Note that the
\textbf{\emph{professor/TA is a resource of last resort}}: only come to
them after you've exhausted all other options.

\hypertarget{use-of-class-materials}{%
\subsubsection{Use of Class Materials}\label{use-of-class-materials}}

Increasingly, with the proliferation of certain websites, questions
about the ownership of course materials have arisen (and Georgetown is
actively working on policies to address these concerns). I consider my
syllabus, lectures, handouts, problem sets, and problem set answers to
be my intellectual property. I respectfully request that you refrain
from sharing my materials in any electronic (or paper) format. You are
welcome to save my lectures for your own use, but they should not be
posted anywhere. Sharing notes, on an occasional basis, with others in
the class is fine as long as they are not posted elsewhere online.
Students found in breach of this policy will fail the course.

\hypertarget{academic-resource-centerdisability-support}{%
\subsubsection{Academic Resource Center/Disability
Support}\label{academic-resource-centerdisability-support}}

If you believe you have a disability, then you should contact the
Academic Resource Center (\url{arc@georgetown.edu}) for further
information. The Center is located in the Leavey Center, Suite 335
(202-687-8354). The Academic Resource Center is the campus office
responsible for reviewing documentation provided by students with
disabilities and for determining reasonable accommodations in accordance
with the Americans with Disabilities Act (ASA) and University policies.
For more information, go to
\url{http://academicsupport.georgetown.edu/disability/}.

\hypertarget{important-academic-policies-and-academic-integrity}{%
\subsubsection{Important Academic Policies and Academic
Integrity}\label{important-academic-policies-and-academic-integrity}}

McCourt School students are expected to uphold the academic policies set
forth by Georgetown University and the Graduate School of Arts and
Sciences. Students should therefore familiarize themselves with all the
rules, regulations, and procedures relevant to their pursuit of a
Graduate School degree. The policies are located
at:'\url{http://grad.georgetown.edu/academics/policies/}

\hypertarget{provosts-policy-accommodating-students-religious-observances}{%
\subsubsection{Provosts Policy Accommodating Students Religious
Observances}\label{provosts-policy-accommodating-students-religious-observances}}

Georgetown University promotes respect for all religions. Any student
who is unable to attend classes or to participate in any examination,
presentation, or assignment on a given day because of the observance of
a major religious holiday (see below) or related travel shall be excused
and provided with the opportunity to make up, without unreasonable
burden, any work that has been missed for this reason and shall not in
any other way be penalized for the absence or rescheduled work. Students
will remain responsible for all assigned work. Students should notify
professors in writing at the beginning of the semester of religious
observances that conflict with their classes. The Office of the Provost,
in consultation with Campus Ministry and the Registrar, will publish,
before classes begin for a given term, a list of major religious
holidays likely to affect Georgetown students. The Provost and the Main
Campus Executive Faculty encourage faculty to accommodate students whose
bona fide religious observances in other ways impede normal
participation in a course. Students who cannot be accommodated should
discuss the matter with an advising dean.

\hypertarget{statement-on-sexual-misconduct}{%
\subsubsection{Statement on Sexual
Misconduct}\label{statement-on-sexual-misconduct}}

Please know that as a faculty member I am committed to supporting
survivors of sexual misconduct, including relationship violence, sexual
harassment and sexual assault. However, university policy also requires
me to report any disclosures about sexual misconduct to the Title IX
Coordinator, whose role is to coordinate the University's response to
sexual misconduct.

Georgetown has a number of fully confidential professional resources who
can provide support and assistance to survivors of sexual assault and
other forms of sexual misconduct. These resources include:

\begin{verbatim}
Associate Director
Jen Schweer, MA, LPC
Health Education Services for Sexual Assault Response and Prevention 
(202) 687-0323
jls242@georgetown.edu
\end{verbatim}

\begin{verbatim}
Erica Shirley
Trauma Specialist
Counseling and Psychiatric Services (CAPS) 
(202) 687-6985
els54@georgetown.edu
\end{verbatim}

More information about campus resources and reporting sexual misconduct
can be found at \url{http://sexualassault.georgetown.edu}.

\hypertarget{course-calendar}{%
\section{Course Calendar}\label{course-calendar}}

\begingroup\fontsize{12}{14}\selectfont

\resizebox{\linewidth}{!}{
\begin{tabu} to \linewidth {>{\centering\arraybackslash}p{.5in}>{\raggedright\arraybackslash}p{1in}>{\raggedright\arraybackslash}p{3in}>{\raggedright}X}
\toprule
Week & Date & Topic & Assignment\\
\midrule
\rowcolor{gray!6}  1 & January 26 & Work Flow and Reproducibility & \\
2 & February 2 & Introduction to Programming in R & \\
\rowcolor{gray!6}  3 & February 9 & Reproducibility in Practice & \\
4 & February 16 & Data Wrangling in R & Problem Set 1 Assigned\\
\rowcolor{gray!6}  5 & February 23 & Data Visualization & Problem Set 1 Due\\
\addlinespace
6 & March 2 & Web Scraping & Problem Set 2 Assigned\\
\rowcolor{gray!6}  7 & March 9 & Geospatial Data & Problem Set 2 Due\\
8 & March 16 & Text as Data & Problem Set 3 Assigned\\
\rowcolor{gray!6}  9 & March 23 & Introduction to Statistical Learning & Problem Set 3 Due\\
- & March 30 & Spring Break; No class & \\
\addlinespace
\rowcolor{gray!6}  10 & April 6 & Applications in Supervised Learning (Regression) & Project Proposal Due; Problem Set 4 Assigned\\
11 & April 13 & Applications in Supervised Learning (Classification) & Problem Set 4 Due\\
\rowcolor{gray!6}  12 & April 20 & Interpretable Machine Learning & Problem Set 5 Assigned\\
13 & April 27 & Applications in Unsupervised Learning & Problem Set 5 Due\\
\rowcolor{gray!6}  14 & May 4 & Project Presentations & \\
\addlinespace
Final & May 18 & Final Project Due (9:00 PM) & \\
\bottomrule
\end{tabu}}
\endgroup{}

\textbf{IMPORTANT: This syllabus is subject to change and may be amended
throughout the course to reflect any changes deemed necessary by the
professor. Any changes will be announced in-class or on Slack.}

\end{document}
