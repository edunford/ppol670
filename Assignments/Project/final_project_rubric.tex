% Options for packages loaded elsewhere
\PassOptionsToPackage{unicode}{hyperref}
\PassOptionsToPackage{hyphens}{url}
%
\documentclass[
  11pt,
]{article}
\usepackage{lmodern}
\usepackage{amssymb,amsmath}
\usepackage{ifxetex,ifluatex}
\ifnum 0\ifxetex 1\fi\ifluatex 1\fi=0 % if pdftex
  \usepackage[T1]{fontenc}
  \usepackage[utf8]{inputenc}
  \usepackage{textcomp} % provide euro and other symbols
\else % if luatex or xetex
  \usepackage{unicode-math}
  \defaultfontfeatures{Scale=MatchLowercase}
  \defaultfontfeatures[\rmfamily]{Ligatures=TeX,Scale=1}
\fi
% Use upquote if available, for straight quotes in verbatim environments
\IfFileExists{upquote.sty}{\usepackage{upquote}}{}
\IfFileExists{microtype.sty}{% use microtype if available
  \usepackage[]{microtype}
  \UseMicrotypeSet[protrusion]{basicmath} % disable protrusion for tt fonts
}{}
\makeatletter
\@ifundefined{KOMAClassName}{% if non-KOMA class
  \IfFileExists{parskip.sty}{%
    \usepackage{parskip}
  }{% else
    \setlength{\parindent}{0pt}
    \setlength{\parskip}{6pt plus 2pt minus 1pt}}
}{% if KOMA class
  \KOMAoptions{parskip=half}}
\makeatother
\usepackage{xcolor}
\IfFileExists{xurl.sty}{\usepackage{xurl}}{} % add URL line breaks if available
\IfFileExists{bookmark.sty}{\usepackage{bookmark}}{\usepackage{hyperref}}
\hypersetup{
  pdfauthor={PPOL670 -- Introduction to Data Science},
  hidelinks,
  pdfcreator={LaTeX via pandoc}}
\urlstyle{same} % disable monospaced font for URLs
\usepackage[margin=1in]{geometry}
\usepackage{graphicx,grffile}
\makeatletter
\def\maxwidth{\ifdim\Gin@nat@width>\linewidth\linewidth\else\Gin@nat@width\fi}
\def\maxheight{\ifdim\Gin@nat@height>\textheight\textheight\else\Gin@nat@height\fi}
\makeatother
% Scale images if necessary, so that they will not overflow the page
% margins by default, and it is still possible to overwrite the defaults
% using explicit options in \includegraphics[width, height, ...]{}
\setkeys{Gin}{width=\maxwidth,height=\maxheight,keepaspectratio}
% Set default figure placement to htbp
\makeatletter
\def\fps@figure{htbp}
\makeatother
\setlength{\emergencystretch}{3em} % prevent overfull lines
\providecommand{\tightlist}{%
  \setlength{\itemsep}{0pt}\setlength{\parskip}{0pt}}
\setcounter{secnumdepth}{-\maxdimen} % remove section numbering

\title{\textbf{Final Project Rubric}}
\author{\Large PPOL670 -- Introduction to Data Science}
\date{Spring 2021}

\begin{document}
\maketitle

\textbf{Student}:\_\_\_\_\_\_\_\_\_\_\_\_\_\_\_\_\_\_\_\_\_\_\_\_\_\_\_\_\_\_\_\_\_\_\_\_\_\_\_\_\_\_\_\_\_

\textbf{Project Name}:
\_\_\_\_\_\_\_\_\_\_\_\_\_\_\_\_\_\_\_\_\_\_\_\_\_\_\_\_\_\_\_\_\_\_\_\_\_\_\_\_

\textbf{Total Score}: \_\_\_\_ / 50

\hypertarget{project-materials}{%
\subsubsection{Project Materials}\label{project-materials}}

\emph{4 points}

\begin{itemize}
\tightlist
\item
  Report was posted to Canvas as a \texttt{.zip} containing the
  following items:

  \begin{itemize}
  \tightlist
  \item
    Report was rendered using RMarkdown as any one of the following file
    types: \texttt{.pdf}, \texttt{.html}, \texttt{.docx}. File was
    titled \texttt{lastname\_firstname\_final\_report.pdf}. (\_\_/1
    point)
  \item
    \texttt{.Rmd} file containing all the code used to generate the
    analytics in the report. File was titled
    \texttt{lastname\_firstname\_final\_report.Rmd}. (\_\_/1 point)
  \item
    Student included the data used in a \texttt{Data/} folder. (\_\_/1
    point)
  \item
    Student included an \texttt{.Rproj}. (\_\_/1 point)
  \end{itemize}
\end{itemize}

\hypertarget{document-presentation}{%
\subsubsection{Document Presentation}\label{document-presentation}}

\emph{16 points}

\begin{itemize}
\tightlist
\item
  \textbf{Student used professional looking visualizations in the
  report:}

  \begin{itemize}
  \tightlist
  \item
    Figures were easy to understand? (\_\_/1 point)
  \item
    Figures made sense within the context of the report? (\_\_/1 point)
  \item
    Student described the purpose and the insight drawn from the figure
    in the text? (\_\_/1 point)
  \item
    Figures referenced in the t ext are labeled, i.e.~references to
    ``figure 1'' correspond to the figure title (e.g.~``Figure 1:
    Title'')? (\_\_/1 point)
  \item
    Figures include titles? (\_\_/1 point)
  \item
    Figures labels/axes/text are readable? (\_\_/1 point)
  \item
    Color scheme made sense; easy to differentiate between colored items
    (\_\_/1 point)
  \item
    Figures were appropriately proportioned to the document? (\_\_/1
    point)
  \end{itemize}
\item
  \textbf{Student used \texttt{R} Markdown for a professional looking
  report:}

  \begin{itemize}
  \tightlist
  \item
    Report was rendered without errors or warnings. (\_\_/1 point)
  \item
    No code was visible in the report. (\_\_/1 point)
  \item
    No raw output was visible in the report. (\_\_/1 point)
  \item
    Report includes a title, author byline, and word count. (\_\_/1
    point)
  \item
    Report is 12 pages in length (double-spaced; 12 pt font; if rendered
    as \texttt{.pdf}/\texttt{.docx}) or 3000 words.\footnote{Note that
      your citations do not count against your word/page count.} (\_\_/1
    point)
  \item
    Report contained no (or few) grammatical/spelling errors. (\_\_/1
    point)
  \item
    Report reads as a single cohesive document. (\_\_/1 point)
  \item
    Student cited academic, data, and package sources. (\_\_/1 point)

    \begin{itemize}
    \tightlist
    \item
      To cite a package, use \texttt{citation("package\_name")} to get a
      the citation information for a package,
      e.g.~\texttt{citation("ggplot2")} will yield ``\emph{H. Wickham.
      ggplot2: Elegant Graphics for Data Analysis. Springer-Verlag New
      York, 2016.}''
    \end{itemize}
  \end{itemize}
\end{itemize}

\hypertarget{content}{%
\subsubsection{Content}\label{content}}

\emph{Points 30}

The student's project sufficiently addressed these general areas.

\begin{itemize}
\item
  \textbf{Introduction} (\_\_/5 point)

  \begin{itemize}
  \item
    Student clearly established the aim of the project.
  \item
    Student offered a clear roadmap of the report (i.e what is covered
    in the report).
  \end{itemize}
\item
  \textbf{Problem Statement and Background} (\_\_/5 point)

  \begin{itemize}
  \item
    Student offered a clear and complete statement of the problem and/or
    aim of their analysis.
  \item
    Student included a brief summary of any related work (i.e.~a
    \emph{light} literature review)
  \end{itemize}
\item
  \textbf{Data} (\_\_/5 point)

  \begin{itemize}
  \item
    Student outlined where their data came from.
  \item
    Student clearly specified:

    \begin{itemize}
    \tightlist
    \item
      the unit of observation;
    \item
      variables of interest;
    \item
      potential issues in the data (e.g.~missingness, coverage, etc.)
    \end{itemize}
  \item
    Student articulate the steps they took to wrangle the data.
  \end{itemize}
\item
  \textbf{Analysis} (\_\_/5 point)

  \begin{itemize}
  \tightlist
  \item
    Student described the methods/tools they explored in their project.

    \begin{itemize}
    \tightlist
    \item
      Justified the tools/methods that they used.
    \item
      Adequately described what the tools/methods are doing.
    \item
      Note: Assume the reader is smart but doesn't know
      \texttt{R}/Machine Learning well. That is, be crystal clear about
      what you're doing and why.
    \end{itemize}
  \end{itemize}
\item
  \textbf{Results} (\_\_/5 point)

  \begin{itemize}
  \item
    Student gave a detailed summary of their results.
  \item
    Student presented their results clearly and concisely.
  \item
    Student used visualizations (and tables) whenever
    possible/appropriate.
  \end{itemize}
\item
  \textbf{Discussion} (\_\_/5 point)

  \begin{itemize}
  \tightlist
  \item
    Student spoke on the ``success'' of their project (as defined in
    their proposal).

    \begin{itemize}
    \tightlist
    \item
      ``Did you achieve what you set out to do? If not why?''
    \end{itemize}
  \item
    Student articulate how they would expand the analysis if given more
    time.
  \end{itemize}
\end{itemize}

\end{document}
