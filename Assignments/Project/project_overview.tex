% Options for packages loaded elsewhere
\PassOptionsToPackage{unicode}{hyperref}
\PassOptionsToPackage{hyphens}{url}
%
\documentclass[
  12pt,
]{article}
\usepackage{lmodern}
\usepackage{amssymb,amsmath}
\usepackage{ifxetex,ifluatex}
\ifnum 0\ifxetex 1\fi\ifluatex 1\fi=0 % if pdftex
  \usepackage[T1]{fontenc}
  \usepackage[utf8]{inputenc}
  \usepackage{textcomp} % provide euro and other symbols
\else % if luatex or xetex
  \usepackage{unicode-math}
  \defaultfontfeatures{Scale=MatchLowercase}
  \defaultfontfeatures[\rmfamily]{Ligatures=TeX,Scale=1}
\fi
% Use upquote if available, for straight quotes in verbatim environments
\IfFileExists{upquote.sty}{\usepackage{upquote}}{}
\IfFileExists{microtype.sty}{% use microtype if available
  \usepackage[]{microtype}
  \UseMicrotypeSet[protrusion]{basicmath} % disable protrusion for tt fonts
}{}
\makeatletter
\@ifundefined{KOMAClassName}{% if non-KOMA class
  \IfFileExists{parskip.sty}{%
    \usepackage{parskip}
  }{% else
    \setlength{\parindent}{0pt}
    \setlength{\parskip}{6pt plus 2pt minus 1pt}}
}{% if KOMA class
  \KOMAoptions{parskip=half}}
\makeatother
\usepackage{xcolor}
\IfFileExists{xurl.sty}{\usepackage{xurl}}{} % add URL line breaks if available
\IfFileExists{bookmark.sty}{\usepackage{bookmark}}{\usepackage{hyperref}}
\hypersetup{
  pdfauthor={PPOL670 -- Introduction to Data Science},
  hidelinks,
  pdfcreator={LaTeX via pandoc}}
\urlstyle{same} % disable monospaced font for URLs
\usepackage[margin=1in]{geometry}
\usepackage{longtable,booktabs}
% Correct order of tables after \paragraph or \subparagraph
\usepackage{etoolbox}
\makeatletter
\patchcmd\longtable{\par}{\if@noskipsec\mbox{}\fi\par}{}{}
\makeatother
% Allow footnotes in longtable head/foot
\IfFileExists{footnotehyper.sty}{\usepackage{footnotehyper}}{\usepackage{footnote}}
\makesavenoteenv{longtable}
\usepackage{graphicx,grffile}
\makeatletter
\def\maxwidth{\ifdim\Gin@nat@width>\linewidth\linewidth\else\Gin@nat@width\fi}
\def\maxheight{\ifdim\Gin@nat@height>\textheight\textheight\else\Gin@nat@height\fi}
\makeatother
% Scale images if necessary, so that they will not overflow the page
% margins by default, and it is still possible to overwrite the defaults
% using explicit options in \includegraphics[width, height, ...]{}
\setkeys{Gin}{width=\maxwidth,height=\maxheight,keepaspectratio}
% Set default figure placement to htbp
\makeatletter
\def\fps@figure{htbp}
\makeatother
\setlength{\emergencystretch}{3em} % prevent overfull lines
\providecommand{\tightlist}{%
  \setlength{\itemsep}{0pt}\setlength{\parskip}{0pt}}
\setcounter{secnumdepth}{-\maxdimen} % remove section numbering
\usepackage{booktabs}
\usepackage{longtable}
\usepackage{array}
\usepackage{multirow}
\usepackage{wrapfig}
\usepackage{float}
\usepackage{colortbl}
\usepackage{pdflscape}
\usepackage{tabu}
\usepackage{threeparttable}
\usepackage{threeparttablex}
\usepackage[normalem]{ulem}
\usepackage{makecell}
\usepackage{xcolor}

\title{\textbf{Project Overview}}
\author{\Large PPOL670 -- Introduction to Data Science}
\date{Spring 2021}

\begin{document}
\maketitle

{
\setcounter{tocdepth}{3}
\tableofcontents
}
\hypertarget{overview}{%
\section{Overview}\label{overview}}

The following provides an overview of the data science project that you
will be responsible for completing by the end of the semester. The
project is an opportunity to apply the skills and tools that you've
learned throughout the course on an area of substantive interest to you.
The aim of this project is to build a model that predicts an outcome of
interest (broadly defined).

The project is composed of three distinct parts: a proposal, a
presentation, and a report. The proposal should outline the general plan
for the project and will serve as an opportunity for the professor and
teaching assistant to provide guidance on its feasibility. The
presentation is an opportunity to present your work in mid-stream to
receive verbal feedback from the professor, TA, and classmates. These
comments will hopefully help you as you move forward with the final
report. The report is the written analysis of the project in its
entirety. The report will be due on May 18 @ 9pm (PPOL670's designated
finals slot).

The proposal, presentation materials, and the report should be generated
using
\href{https://rmarkdown.rstudio.com/authoring_quick_tour.html}{RMarkdown}
and should follow all reproducibility practices discussed in class.

\hypertarget{project-proposal}{%
\section{Project Proposal}\label{project-proposal}}

\begin{longtable}[]{@{}lll@{}}
\toprule
\textbf{Due} & \textbf{Proportion of Grade} &
\textbf{Length}\tabularnewline
\midrule
\endhead
April 13 & 5\% & 750-1000 words\tabularnewline
\bottomrule
\end{longtable}

The project proposal asks that you sketch out a general project proposal
(750-1000 words). The proposal should offer the following information:

\begin{enumerate}
\def\labelenumi{\arabic{enumi}.}
\tightlist
\item
  A high-level statement of the problem you intend to address or the
  analysis you aim to generate;

  \begin{itemize}
  \tightlist
  \item
    What is the research question/problem?
  \item
    Have others studied this question/problem before? If so, what are
    some conclusions they've drawn. What is still unknown?
  \end{itemize}
\item
  What is the outcome variable that you're looking to explore/predict?

  \begin{itemize}
  \tightlist
  \item
    Describe the variable and how it relates conceptually to the
    research question/problem.
  \item
    Provide summary statistics of the outcome variable.

    \begin{itemize}
    \tightlist
    \item
      Five number summary, plot the distribution of the variable, and/or
      any other valuable summaries of the variable (e.g.~plot the
      spatial distribution of the variable)
    \end{itemize}
  \item
    Is there missingness in the variable? If so, explore what
    observations are missing data. Is there something systematic about
    that missingness (e.g.~``a majority of authoritarian countries are
    missing in the data'')
  \item
    Note: \emph{The outcome variable cannot be pulled from the World
    Bank, United Nations, International Monetary Fund, or Our World in
    Data without explicit permission from the instructor}.
  \end{itemize}
\item
  What predictor variables do you plan on using to model the outcome?

  \begin{itemize}
  \tightlist
  \item
    You should have a minimum of 10 predictor variables in your model.
  \item
    Outline where you plan to get these data.
  \item
    \emph{No more than 20\% of your predictor variables} should be from
    the World Bank, United Nations, International Monetary Fund, or Our
    World in Data.
  \item
    You do not need to describe these variables at this point in time.
    Rather I am looking that for a clear plan to get these data. If you
    have these, you can offer some brief descriptive statistics.
  \end{itemize}
\item
  A definition for what ``success'' means with respect to your project.

  \begin{itemize}
  \tightlist
  \item
    In your words, what would a successful project look like? How will
    you know that you solved the problem or accomplished your goal?
  \item
    Four weeks isn't a long time to complete a project like this.
    Thinking serious about what a ``finished'' or ``successful'' project
    might look like. This will help you set realistic
    goals/expectations.
  \end{itemize}
\end{enumerate}

Please be detailed but \emph{succinct} as possible when writing.
\emph{Any material that exceeds 1000 words will not be considered when
grading/reading}. There is no advantage/incentive to exceeding the word
limit. Be sure to properly cite any referenced materials and/or packages
(it is okay if your work cited runs over the word limit. Your work cited
will not be considered in the word count).

\hypertarget{project-presentation}{%
\subsection{Project Presentation}\label{project-presentation}}

\begin{longtable}[]{@{}lll@{}}
\toprule
\textbf{Due} & \textbf{Proportion of Grade} &
\textbf{Length}\tabularnewline
\midrule
\endhead
May 4 & 10\% & 7 minutes in length\tabularnewline
\bottomrule
\end{longtable}

\hypertarget{part-1-video-presentation}{%
\subsection{Part 1: Video
Presentation}\label{part-1-video-presentation}}

Please prepare and record a 7 minute presentation that walks us through
the progress you've made on your project to date. The presentation is an
opportunity to summarize your project and talk through your
(preliminary) results. Moreover, it will provide an opportunity for both
your peers and the Professor/TA/classmates to provide constructive
feedback, which you can then incorporate into your final paper.

When preparing your recording, please prepare slides \emph{using
\texttt{R\ Markdown}}. \textbf{Students should not ``live code'' or show
output from their computer.} This is meant to be a polished presentation
as if you were giving it in-person.

The slides should generally adhere to the following format. You should
plan on having up to 5-10 slides in total. The layout of the
presentation should take on the following form.

\begin{enumerate}
\def\labelenumi{\arabic{enumi}.}
\item
  (1-3 slides) Problem statement and Background
\item
  (1-3 slides) Methods you explored or considered using.
\item
  (1-3 slides) The methods/tools you used, and the rationale for their
  use.
\item
  (2-4 slides) Results (however preliminary).

  \begin{itemize}
  \tightlist
  \item
    Show main visuals, analyses/tables, and/or any products built
    (interactive graphics, websites, etc.)
  \end{itemize}
\item
  (1-2 slides) Lessons learned thus far and/or plans to mitigate
  challenges.
\end{enumerate}

Students must submit both their slides \emph{and} the \texttt{.rmd} file
used to render the slides along with their video recording as a
\texttt{.zip} file to CANVAS by the end of the scheduled class time.
There will be no in-person and/or virtual class meeting this day.

\begin{quote}
\emph{Note that it is vital that all students submit their video on time
so that others will have sufficient time to provide feedback.}
\end{quote}

\hypertarget{part-2-feedback}{%
\subsection{Part 2: Feedback}\label{part-2-feedback}}

Each student will be randomly assigned the names of 5 peers in their
class. The names will be circulated on \textbf{May 4}. Each student will
be required to watch the recordings of their assigned classmates and
provide substantive feedback by \textbf{Sunday May 9 11:59PM}. All
comments/Feedback should be written on a shared \textbf{Google
Document}, which will be circulated on \textbf{May 4} via the class
Slack channel. The recorded presentations will be stored in a share
folder on CANVAS. All enrolled students will have access to this folder.

\hypertarget{project-report}{%
\section{Project Report}\label{project-report}}

\begin{longtable}[]{@{}lll@{}}
\toprule
\textbf{Due} & \textbf{Proportion of Grade} &
\textbf{Length}\tabularnewline
\midrule
\endhead
May 18 & 30\% & 3000 words\tabularnewline
\bottomrule
\end{longtable}

The report is a complete description of the project's analysis and
results. The report should be 3000 words in length and cover the below
bullet points. Note that your work cited can exceed the 3000 words
limit. Below I've outlined points that one should aim to discuss in each
section. Note that paper should read as a cohesive report, so do not
respond to these bullet points verbatim.

\begin{itemize}
\item
  \textbf{Introduction}

  \begin{itemize}
  \tightlist
  \item
    What is the aim of the project?

    \begin{itemize}
    \tightlist
    \item
      Summarize the problem
    \item
      State your goals
    \end{itemize}
  \item
    What do you do in this report?

    \begin{itemize}
    \tightlist
    \item
      offer a roadmap of the project
    \end{itemize}
  \end{itemize}
\item
  \textbf{Problem Statement and Background}

  \begin{itemize}
  \item
    Give a clear and complete statement of the problem and/or aim of
    your analysis.
  \item
    Include a brief summary of any related work that has tried to tackle
    a project similar to yours (i.e.~a \emph{light} literature review)
  \end{itemize}
\item
  \textbf{Data}

  \begin{itemize}
  \item
    What is the unit of observation?
  \item
    What is the outcome variable?

    \begin{itemize}
    \tightlist
    \item
      How is it measured?
    \item
      Where does it come from?
    \item
      Please describe how the outcome variable is distributed using a
      table and/or graph.
    \end{itemize}
  \item
    What are your predictor variables?

    \begin{itemize}
    \tightlist
    \item
      How are they measured?
    \item
      Where do they come from?
    \item
      Please describe how the predictor variable are distributed using a
      table and/or graph.
    \end{itemize}
  \item
    Outline any potential issues with the data:

    \begin{itemize}
    \tightlist
    \item
      Missingness
    \item
      Lack of variation and/or availibility
    \item
      Any potential sources of bias
    \end{itemize}
  \item
    How do you overcome/mitigate these issues in your analysis?
  \end{itemize}
\item
  \textbf{Analysis}

  \begin{itemize}
  \item
    Describe the methods/tools you explored in your project.
  \item
    Outline in detail our entire analysis.

    \begin{itemize}
    \tightlist
    \item
      Justify the tools/methods that you used.
    \item
      Assume the reader is smart but doesn't know \texttt{R}/Machine
      Learning well. That is, be crystal clear about what you're doing
      and why.
    \end{itemize}
  \item
    Note that this section should walk us through what you're doing and
    how you're planning on doing it. There should be no results
    presented in this section.
  \end{itemize}
\item
  \textbf{Results}

  \begin{itemize}
  \item
    Give a detailed summary of your results. Present your results
    clearly and concisely.
  \item
    Please use visualizations and tables whenever possible.
  \item
    Be sure to:

    \begin{itemize}
    \tightlist
    \item
      Discuss the performance of your predictive model.
    \item
      Use interpretable machine learning to talk about which variables
      were important in the prediction task and how they relate to the
      outcome (i.e.~PDP/ICE/Surrogate Models)
    \end{itemize}
  \end{itemize}
\item
  \textbf{Discussion}

  \begin{itemize}
  \item
    What conclusions should we pull from your analysis?
  \item
    What are the limitations (i.e.~what can't we say given your
    findings)?
  \item
    How would you expand the analysis if given more time?
  \item
    Speak on the ``success'' of your project (as defined in your
    proposal).

    \begin{itemize}
    \tightlist
    \item
      Did you achieve what you set out to do? If not why?
    \end{itemize}
  \end{itemize}
\end{itemize}

The reports must be submitted as a hardcopy (i.e.~the \texttt{.rmd}
notebook must be rendered as a \texttt{.html}) to CANVAS by 9PM on May
18th. \textbf{\emph{Note that given the page constraints, no \texttt{R}
code should be visible in the rendered document.}}

\end{document}
